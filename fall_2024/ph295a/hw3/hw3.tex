\documentclass{article}
\usepackage[margin=1in]{geometry}
\usepackage{amsmath}
\usepackage{amssymb}
\usepackage{enumitem}
\usepackage{graphicx}
\usepackage{physics}
\usepackage{amsmath}
\usepackage{amssymb}

\begin{document}

\title{Problem Set 3}
\author{Your Name}
\date{October 14, 2024}
\maketitle

\section{Problem 1: Girvin and Yang, 7.17}

A certain 1D solid aligned in the x direction consists of “S” atoms and “P” atoms in an infinite alternating sequence SPSPSP. The S atoms have a single atomic s-orbital energy $\epsilon_s$. The P atoms have $p_x$, $p_y$, and $p_z$ orbitals of energy $\epsilon_p$. In a nearest-neighbor tight-binding model, the transfer integrals from an S-atom to the P-atom on its right are $J_x$, $J_y$, and $J_z$.\\
\subsection{}
(a) What are the corresponding transfer integrals to the P-atom on the left?\\
\subsubsection{Solution}
If the chain is aligned in the x direction, then we know that the lobes of the $p_x$ orbital are aligned in the x direction and, thus, have opposite signs on the left and right. The lobes of the $p_y$ and $p_z$ orbitals are aligned in the y and z directions, respectively, and they don't care that the chain is aligned in the x direction. Thus, their transfer integrals will be unchanged with the different direction, but that of the $p_x$ orbital will change sign. Therefore, the transfer integrals to the P-atom on the left are $-J_x$, $J_y$, and $J_z$.
\subsection{}
(b) Which of the $J$'s are zero by symmetry?\\
\subsubsection{Solution}
The $J_y$ and $J_z$ transfer integrals are zero by symmetry. Because the $p_y$ and $p_z$ orbitals are aligned not in the x direction, there will be an equal amplitude of coupling from the positive lobe to the S atom and the negative lobe to the S atom, and so these will cancel out perfectly by symmetry.
\subsection{}
(c) Compute the band structure and sketch the energy levels.\\
\subsubsection{Solution}
Consider that we will have 4 basis functions in our Hamiltonian, $\ket{S}$, $\ket{P_x}$, $\ket{P_y}$, and $\ket{P_z}$ where we know that the matrix elements will be $\bra{S}H\ket{S}=\epsilon_s$, $\bra{P_x}H\ket{P_x}=\epsilon_p$, $\bra{P_y}H\ket{P_y}=\epsilon_p$, and $\bra{P_z}H\ket{P_z}=\epsilon_p$. We also know that the transfer integrals for hopping from $\ket{S}$ to both $\ket{P_y}$ and $\ket{P_z}$ are zero, so we need only to determine $\bra{S}H\ket{P_x}$ and $\bra{P_x}H\ket{S}$. So our Hamiltonian will look like:
\begin{equation}
    H = \begin{pmatrix}
    \epsilon_s & \bra{S}H\ket{P_x} & 0 & 0\\
    \bra{P_x}H\ket{S} & \epsilon_p & 0 & 0\\
    0 & 0 & \epsilon_p & 0\\
    0 & 0 & 0 & \epsilon_p 
    \end{pmatrix}
\end{equation}
and we want to evaluate $\bra{S}H\ket{P_x}$ and $\bra{P_x}H\ket{S}$ in k-space to determine the band structure. We define the Bloch functions as:
\begin{equation}
    \ket{S_k} = \frac{1}{\sqrt{N}}\sum_{j}e^{-ikna}\ket{S_n}
\end{equation}
and
\begin{equation}
    \ket{P_{x,k}} = \frac{1}{\sqrt{N}}\sum_{j}e^{-ikna}\ket{P_{x,n}}
\end{equation}
where $N$ is the number of atoms in the chain and $na$ is the position of the $n$th unit cell with that is constant $a$ so that we know the positions of the S atoms are $na$ while those of the P atoms are $(n+1/2)a$. First, we want to evaluate
\begin{equation}
    \bra{S_k}H\ket{P_{x,k}} = \frac{1}{N}\sum_{n,m}e^{ik(n-m)a}\bra{S_n}H\ket{P_{x,m}}
\end{equation}
Now, note we have the constraint that $m=n\pm\frac{1}{2}$, which implies that
\begin{equation}
    \bra{S_n}H\ket{P_{x,m}} = \begin{cases}
    \bra{S_n}H\ket{P_{x,n+1/2}}=+J_x & \text{if } m=n+1/2\\
    \bra{S_n}H\ket{P_{x,n-1/2}}=-J_x& \text{if } m=n-1/2
    \end{cases}
\end{equation}
So we can get rid of the index $m$ in the sum and rewrite the sum as
\begin{align}
    \bra{S_k}H\ket{P_{x,k}} &= \frac{1}{N}\sum_{n}e^{ik(n-(n+1/2))a}\bra{S_n}H\ket{P_{x,n+1/2}} + e^{ik(n-(n-1/2))a}\bra{S_n}H\ket{P_{x,n-1/2}} \\
    &= \frac{1}{N}\sum_{n}e^{-ika/2}J_x - e^{ika/2}(J_x)\\
    &= \frac{N}{N}J_x(e^{-ika/2}-e^{ika/2})\\
    &= J_x(e^{-ika/2}-e^{ika/2})\\
\end{align}
Euler's formula tells us that $e^{i\theta}-e^{-i\theta}=2i\sin(\theta)$, so we can rewrite the sum as
\begin{equation}
    \bra{S_k}H\ket{P_{x,k}} = -2iJ_x\sin(ka/2)
\end{equation}
Now we want to evaluate $\bra{P_{x,k}}H\ket{S_k}$, but we know that this is just the complex conjugate of what we already found, so we know that
\begin{equation}
    \bra{P_{x,k}}H\ket{S_k} = 2iJ_x\sin(ka/2)
\end{equation}
So we can rewrite our Hamiltonian as
\begin{equation}
    H = \begin{pmatrix}
    \epsilon_s & -2iJ_x\sin(ka/2) & 0 & 0\\
    2iJ_x\sin(ka/2) & \epsilon_p & 0 & 0\\
    0 & 0 & \epsilon_p & 0\\
    0 & 0 & 0 & \epsilon_p 
    \end{pmatrix}
\end{equation}
In order to diagonalize this super matrix, we find where the determinant generated by the characteristic equation is zero.
\begin{equation}
    \begin{pmatrix}
    \epsilon_s-E & -2iJ_x\sin(ka/2)\\
    2iJ_x\sin(ka/2) & \epsilon_p-E
    \end{pmatrix}
\end{equation}
Sympy gives roots of
\begin{equation}
    E(k) = \frac{\epsilon_{p}}{2} + \frac{\epsilon_{s}}{2} \ \frac{\sqrt{16 J_{x}^{2} \sin^{2}{\left(\frac{a k}{2} \right)} + \epsilon_{p}^{2} - 2 \epsilon_{p} \epsilon_{s} + \epsilon_{s}^{2}}}{2}
\end{equation}
I have a fine motor impairment so not able to sketch out the bands, but theoretically, we could just plug in different values for $k$ and see how the energy levels change.\\

\subsection{}
(d) Why is there a total of four bands, and why are some of them dispersionless?\\
\subsubsection{Solution}
There are four bands because there are four basis functions in the Hamiltonian. As explained earlier, 2 of them correspond to $\ket{P_y}$ and $\ket{P_z}$, which have no transfer integrals to the S atoms, so they are dispersionless. The other two bands correspond to a mixing of the $\ket{S}$ and $\ket{P_x}$ orbitals, for which we just computed the dispersion relation.\\
\subsection{}
(e) Suppose the P atom contains a spin–orbit coupling of the form $\Delta H = \Gamma  \mathbf{L} \cdot \mathbf{S}$, where $\mathbf{L}$ is the orbital angular momentum and $\mathbf{S}$ is the spin operator. What are the energy levels and eigenstates of an isolated P atom? Describe how you would solve for the tight-binding band structure in this case. Write down all the transfer matrix elements, but do not solve the Hamiltonian.
\subsubsection{Solution}
We start by just considering the isolated P atom. We know that the energy levels of the P atom are given by
\begin{equation}
    H_p = \epsilon_p + H_{SO}
\end{equation}
where $H_{SO} = \Gamma \mathbf{L} \cdot \mathbf{S}$. We know that the angular momentum quantum number of a p orbital is $l=1$, so we know that the possible values of $m_l$ are $-1, 0, 1$. We also know that the spin quantum number of an electron is $s=1/2$, so the possible values of $m_s$ are $-1/2, 1/2$. This suggests that for and isolated P atom, we have a basis of 6 states, $\ket{1, -1/2}$, $\ket{1, 1/2}$, $\ket{0, -1/2}$, $\ket{0, 1/2}$, $\ket{-1, -1/2}$, $\ket{-1, 1/2}$. Now recall the idemtity $\mathbf{L} \cdot \mathbf{S} = \frac{1}{2}(J^2 - L^2 - S^2)$, where $J$ is the total angular momentum. We know that $J = L + S$, so its quantum  number $j$ can take on values $l+s, l+s-1, ..., |l-s|$. This suggests that the possible values of $j$ are $3/2, 1/2$. Thus, we can compute energies of the states $\ket{j}$ by
\begin{align}
    E_{j} &= \epsilon_p + \Gamma \frac{1}{2} \left(j(j+1) - 1(1+1) - 1/2(1/2+1) \right) \implies\\
    E_{3/2} &= \epsilon_p + \Gamma \frac{1}{2} \left(\frac{15}{4} - 2 - 3/4 \right) = \epsilon_p + \frac{1}{2}\Gamma\\
    E_{1/2} &= \epsilon_p + \Gamma \frac{1}{2} \left(\frac{3}{4} - 2 - 3/4 \right) = \epsilon_p - \Gamma
\end{align}




\section{Problem 2: SSH Chain Eigenvalues and Eigenvectors}

Write a Mathematica (or Python) program to determine the eigenvectors and eigenvalues of the SSH chain with open boundary conditions (but no defects in the bulk). Discuss how the existence of nearly zero energy states localized on an edge depends upon the relative values of $t_1$ and $t_2$, and on whether the number of sites is even or odd.\\

\section{Problem 3: Girvin and Yang, 7.18}

\textbf{Problem:} \\
We know that eigenfunctions of a single-electron Hamiltonian with a rotationally invariant potential take the form of states of fixed $n$ and $l$ but different $m$ being degenerate. Now consider adding a perturbing potential of the form:
\[
V(x, y, z) = \lambda(x^4 + y^4 + z^4),
\]
which breaks full rotation symmetry but respects cubic symmetry. That is, the symmetry of the system has been reduced from spherical to cubic by $V$. Treat $V$ to first order perturbatively for the set of degenerate states with $l=2$ and any $n$ (but assume no accidental degeneracy for different $n$'s), and show that the five-fold degeneracy is split into two levels, with three- and two-fold degeneracies, respectively. The angular parts of the states in each group take the forms of the $t_{2g}$ and $e_g$ orbitals, or any linear combinations within each group.
\subsubsection{Solution}
The first thing we do is to write down the 5-fold degenerate states in the $l=2$ manifold. For convenience, we will just deal with linear combinations of the spherical harmonics, which respond to the classically defined d orbitals:
\begin{align}
    d_{xy} &= \frac{1}{\sqrt{2}}(Y_{2, 2} + Y_{2, -2}) = \sqrt{\frac{15}{8\pi}}\sin^2(\theta)\sin(2\phi)\\
    d_{yz} &= \frac{1}{\sqrt{2i}}(Y_{2, 1} - Y_{2, -1}) = \sqrt{\frac{15}{8\pi}}\sin(\theta)\cos(\theta)\sin(\phi)\\
    d_{xz} &= \frac{1}{\sqrt{2}}(Y_{2, 1} + Y_{2, -1}) = \sqrt{\frac{15}{8\pi}}\sin(\theta)\cos(\theta)\cos(\phi)\\
    d_{x^2-y^2} &= \frac{1}{\sqrt{2}}(Y_{2, 2} - Y_{2, -2}) = \sqrt{\frac{15}{8\pi}}\sin^2(\theta)\cos(2\phi)\\
    d_{z^2} &= Y_{2, 0} = \sqrt{\frac{5}{16\pi}}(3\cos^2(\theta) - 1)
\end{align}
Now we want to compute the matrix elements of the perturbing potential $V$ in the basis of these states. We know that the matrix elements of $V$ to first order in the perturbation are given by
\begin{equation}
    \bra{d_{\alpha}}V\ket{d_{\beta}} = \lambda\int d\Omega d_{\alpha}^*\left(x^4 + y^4 + z^4\right)d_{\beta}
\end{equation}
 We will aval with the integals through group theoretic arguments. We now that the $d_{xy}$, $d_{yz}$, and $d_{xz}$ orbitals transform as the $t_{2g}$ representation of the cubic group, while the $d_{x^2-y^2}$ and $d_{z^2}$ orbitals transform as the $e_g$ representation. We are told that the perturbation potential is cubical symmetric so it will transform according to the totally symmetric irrep $A_{1g}$. Now, the product of an irrep with a totally symmetric irrep is just the original irrep, so we know that the matrix elements of $V$ will be diagonal in the $t_{2g}$ and $e_g$ basis, but then we also know that the integral must contain the totally symmetric irrep two be non zero, so we know that the matrix elements of $V$ will be zero between the $t_{2g}$ and $e_g$ orbitals and we are left with 2 sub-matrices:
\begin{equation}
    \begin{pmatrix}
    \bra{d_{xy}}V\ket{d_{xy}} & \bra{d_{xy}}V\ket{d_{yz}} & \bra{d_{xy}}V\ket{d_{xz}}\\
    \bra{d_{yz}}V\ket{d_{xy}} & \bra{d_{yz}}V\ket{d_{yz}} & \bra{d_{yz}}V\ket{d_{xz}}\\
    \bra{d_{xz}}V\ket{d_{xy}} & \bra{d_{xz}}V\ket{d_{yz}} & \bra{d_{xz}}V\ket{d_{xz}}
    \end{pmatrix}
\end{equation}
and
\begin{equation}
    \begin{pmatrix}
    \bra{d_{x^2-y^2}}V\ket{d_{x^2-y^2}} & \bra{d_{x^2-y^2}}V\ket{d_{z^2}}\\
    \bra{d_{z^2}}V\ket{d_{x^2-y^2}} & \bra{d_{z^2}}V\ket{d_{z^2}}
    \end{pmatrix}
\end{equation}
Now, the of diagonal elements in these matrices will vanish because the eigenfunctions are orthogonal.

\section*{Problem 4: Ashcroft \& Mermin, Chapter 10, Problem 2}

In dealing with cubic crystals, the most convenient linear combinations of three degenerate atomic $p$-levels have the form $x \phi(r), y \phi(r)$, and $z \phi(r)$, where the function $\phi$ depends only on the magnitude of the vector $\mathbf{r}$. The energies of the three corresponding $p$-bands are found from (10.12) by setting to zero the determinant


\begin{equation*}
\left|\left(\varepsilon(\mathbf{k})-E_{p}\right) \delta_{i j}+\beta_{i j}+\tilde{\gamma}_{i j}(\mathbf{k})\right|=0 \tag{10.30}
\end{equation*}


where


\begin{align*}
\tilde{\gamma}_{i j}(\mathbf{k}) & =\sum_{\mathbf{R}} e^{i \mathbf{k} \cdot \mathbf{R}} \gamma_{i j}(\mathbf{R}) \\
\gamma_{i j}(\mathbf{R}) & =-\int d \mathbf{r} \psi_{i}^{*}(\mathbf{r}) \psi_{j}(\mathbf{r}-\mathbf{R}) \Delta U(\mathbf{r}) \\
\beta_{i j} & =\gamma_{i j}(\mathbf{R}=0) \tag{10.31}
\end{align*}


(A term multiplying $\delta(\mathbf{k})-E_{p}$, which gives rise to very small corrections analogous to those given by the denominator of ( 10.15 ) in the $s$-band case, has been omitted from (10.30).)\\
\subsection{}
As a consequence of cubic symmetry, show that

\begin{align*}
& \beta_{x x}=\beta_{y y}=\beta_{z z}=\beta \\
& \beta_{x y}=0 \tag{10.32}
\end{align*}
\subsubsection{Solution}
We want to consider the $\beta_{i j}$ terms, which are given by
\begin{equation}
    \beta_{i j} = \gamma_{i j}(\mathbf{R}=0) = -\int d \mathbf{r} \psi_{i}^{*}(\mathbf{r}) \psi_{j}(\mathbf{r}) \Delta U(\mathbf{r})
\end{equation}
First, let us compute $\beta_{x x}$. We know $\psi_{x} = x\phi(r)$, so we can write
\begin{equation}
    \beta_{x x} = -\int d \mathbf{r} \left(x\phi(r)\right)^{*} x\phi(r) \Delta U(\mathbf{r}) = -\int d \mathbf{r} x^2\phi^2(r) \Delta U(\mathbf{r})
\end{equation}
and we can do the same for $\beta_{y y}$ and $\beta_{z z}$, finding:
\begin{align}
    \beta_{y y} &= -\int d \mathbf{r} y^2\phi^2(r) \Delta U(\mathbf{r})\\
    \beta_{z z} &= -\int d \mathbf{r} z^2\phi^2(r) \Delta U(\mathbf{r})
\end{align}
Because of the cubic symmetry, we know that the value of all these integrals will be the same, so we have $\beta_{x x} = \beta_{y y} = \beta_{z z} = \beta$. Now we want to compute $\beta_{x y}$. We know that $\psi_{x} = x\phi(r)$ and $\psi_{y} = y\phi(r)$, so we can write
\begin{equation}
    \beta_{x y} = -\int d \mathbf{r} x\phi(r) y\phi(r) \Delta U(\mathbf{r}) = -\int d \mathbf{r} xy\phi^2(r) \Delta U(\mathbf{r})
\end{equation}
Now everything except the function $xy$ is symmetric in the cubic system, so that integrand is an odd function and the integral will vanish. Thus, we have $\beta_{x y} = 0$.
\subsection{}
(b) Assuming that the $\gamma_{i j}(\mathbf{R})$ are negligible except for nearest-neighbor $\mathbf{R}$, show that $\tilde{\gamma}_{i j}(\mathbf{k})$ is diagonal for a simple cubic Bravais lattice, so that $x \phi(r), y \phi(r)$, and $z \phi(r)$ each generate independent bands. (Note that this ceases to be the case if the $\gamma_{i j}(\mathbf{R})$ for next nearest-neighbor $\mathbf{R}$ are also retained.)\\
\subsubsection{Solution}
Working backwards, in order to show that there are independent bands generated, we want to show that $\tilde{\gamma}_{i j}(\mathbf{k})$ is diagonal. We know that $\tilde{\gamma}_{i j}(\mathbf{k})$ is given by
\begin{equation}
    \tilde{\gamma}_{i j}(\mathbf{k}) = \sum_{\mathbf{R}} e^{i \mathbf{k} \cdot \mathbf{R}} \gamma_{i j}(\mathbf{R})
\end{equation}
So we want to show that $\gamma_{i j}(\mathbf{R})$ is zero for all $i\neq j$ and non-zero for $i=j$. We know that $\gamma_{i j}(\mathbf{R})$ is given by
\begin{equation}
    \gamma_{i j}(\mathbf{R}) = -\int d \mathbf{r} \psi_{i}^{*}(\mathbf{r}) \psi_{j}(\mathbf{r}-\mathbf{R}) \Delta U(\mathbf{r})
\end{equation}
We are told that this integral is negligible except for nearest neighbors, so we deduce that the potential $\Delta U(\mathbf{r})$ is only non-negligible at $\mathbf{r}=0$. So, we need the wave functions in the integrand to overlap at the origin. If we think geometrically, we will only see overlap with the $p_x$ orbital if the neighbor has its lobes pointing in the $x$ direction, which only happens when it is also a $p_x$ orbital. A generalization of this will suggest that we only see a nonzero integral when $i=j$, so we have $\gamma_{i j}(\mathbf{R}) = 0$ for $i\neq j$ and $\gamma_{i j}(\mathbf{R}) \neq 0$ for $i=j$. So, we have proved the necessary condition.
\subsection{}
For a face-centered cubic (fcc) Bravais lattice with only nearest-neighbor \(\gamma_{ij}\) appreciable, show that the energy bands are given by the roots of

\[
0 = \left| \begin{array}{ccc}
\varepsilon(\mathbf{k}) - \varepsilon^{0}(\mathbf{k}) + 4 \gamma_{0} \cos \left( \dfrac{1}{2} k_{y} a \right) \cos \left( \dfrac{1}{2} k_{z} a \right) & -4 \gamma_{1} \sin \left( \dfrac{1}{2} k_{x} a \right) \sin \left( \dfrac{1}{2} k_{y} a \right) & -4 \gamma_{1} \sin \left( \dfrac{1}{2} k_{x} a \right) \sin \left( \dfrac{1}{2} k_{z} a \right) \\[2ex]
-4 \gamma_{1} \sin \left( \dfrac{1}{2} k_{y} a \right) \sin \left( \dfrac{1}{2} k_{x} a \right) & \varepsilon(\mathbf{k}) - \varepsilon^{0}(\mathbf{k}) + 4 \gamma_{0} \cos \left( \dfrac{1}{2} k_{z} a \right) \cos \left( \dfrac{1}{2} k_{x} a \right) & -4 \gamma_{1} \sin \left( \dfrac{1}{2} k_{y} a \right) \sin \left( \dfrac{1}{2} k_{z} a \right) \\[2ex]
-4 \gamma_{1} \sin \left( \dfrac{1}{2} k_{z} a \right) \sin \left( \dfrac{1}{2} k_{x} a \right) & -4 \gamma_{1} \sin \left( \dfrac{1}{2} k_{z} a \right) \sin \left( \dfrac{1}{2} k_{y} a \right) & \varepsilon(\mathbf{k}) - \varepsilon^{0}(\mathbf{k}) + 4 \gamma_{0} \cos \left( \dfrac{1}{2} k_{x} a \right) \cos \left( \dfrac{1}{2} k_{y} a \right)
\end{array} \right| \tag{10.33}
\]

where

\begin{align*}
\varepsilon^{0}(\mathbf{k}) &= E_{p} - \beta \\
&\quad - 4 \gamma_{2} \left[ \cos \left( \dfrac{1}{2} k_{x} a \right) \cos \left( \dfrac{1}{2} k_{z} a \right) + \cos \left( \dfrac{1}{2} k_{x} a \right) \cos \left( \dfrac{1}{2} k_{y} a \right) + \cos \left( \dfrac{1}{2} k_{y} a \right) \cos \left( \dfrac{1}{2} k_{z} a \right) \right], \\[2ex]
\gamma_{0} &= -\int d\mathbf{r} \left[x^2-y\left(y-\frac{1}{2}a\right)\right] \phi(r) \phi\left(\sqrt{x^2+\left(y-\frac{1}{2}a\right)^2+\left(z-\frac{1}{2}a\right)^2}\right) \Delta U(\mathbf{r}), \\[2ex]
\gamma_{1} &= -\int d\mathbf{r} \, x \left( y - \dfrac{1}{2} a \right) \, \phi(r) \, \phi\left( \sqrt{ \left( x - \dfrac{1}{2} a \right)^{2} + \left( y - \dfrac{1}{2} a \right)^{2} + z^{2} } \, \right) \Delta U(\mathbf{r}), \\[2ex]
\gamma_{2} &= -\int d\mathbf{r} \, x \left( x - \dfrac{1}{2} a \right) \, \phi(r) \, \phi\left( \sqrt{ \left( x - \dfrac{1}{2} a \right)^{2} + \left( y - \dfrac{1}{2} a \right)^{2} + z^{2} } \, \right) \Delta U(\mathbf{r}). \tag{10.34}
\end{align*}

\subsubsection{Solution}
The equation $|\varepsilon(\mathbf{k})-E_{p}\delta_{i j}+\beta_{i j}+\tilde{\gamma}_{i j}(\mathbf{k})|=0$ where $i,j=x,y,z$ suggests that we have a 3x3 matrix. From the first part we know that $\beta_{i j} = \beta$ for $i=j$ and $\beta_{i j} = 0$ for $i\neq j$. Now we need to determine the form of $\tilde{\gamma}_{i j}(\mathbf{k})$. We know that $\tilde{\gamma}_{i j}(\mathbf{k}) = \sum_{\mathbf{R}} e^{i \mathbf{k} \cdot \mathbf{R}} \gamma_{i j}(\mathbf{R})$. We are told that the $\gamma_{i j}(\mathbf{R})$ are negligible except for nearest-neighbor $\mathbf{R}$ and the nearest neighbors of a face-centered cubic Bravais lattice are at $\mathbf{R}_1 = \frac{a}{2}\left(\pm x \pm y\right)$, $\mathbf{R}_2 = \frac{a}{2}\left(\pm x \pm z\right)$, and $\mathbf{R}_3 = \frac{a}{2}\left(\pm y \pm z\right)$, so 12 possibilites.
\begin{align}
    \tilde{\gamma}_{ij}(\mathbf{k}) &= \sum_{\mathbf{R}} e^{i \mathbf{k} \cdot \mathbf{R}} \gamma_{i j}(\mathbf{R})\\
    &= e^{i \mathbf{k} \cdot \left(\frac{a}{2}\left(\pm x \pm y\right)\right)} \gamma_{i j}(\frac{a}{2}\left(\pm x \pm y\right)) + e^{i \mathbf{k} \cdot \left(\frac{a}{2}\left(\pm x \pm z\right)\right)} \gamma_{i j}(\frac{a}{2}\left(\pm x \pm z\right)) + e^{i \mathbf{k} \cdot \left(\frac{a}{2}\left(\pm y \pm z\right)\right)} \gamma_{i j}(\frac{a}{2}\left(\pm y \pm z\right))
\end{align}
With the formula:
\begin{equation}
    \gamma_{i j}(\mathbf{R}) = -\int d \mathbf{r} \psi_{i}^{*}(\mathbf{r}) \psi_{j}(\mathbf{r}-\mathbf{R}) \Delta U(\mathbf{r})
\end{equation}
Let's start by computing $\gamma_{xy}(\frac{a}{2}\left(x+y\right))$. We know that $\psi_{x}\left(\mathbf{r}\right) = x\phi(r)$ and 
\begin{equation}
\psi_{y}\left(\mathbf{r}-\frac{a}{2}\left(x+y\right)\right) = \left(y-R_y\right)\phi\left(\mathbf{r}-\frac{a}{2}\left(x+y\right)\right) = \left(y-\frac{a}{2}\right) \phi\left(\sqrt{\left(x-\frac{a}{2}\right)^2+\left(y-\frac{a}{2}\right)^2+z^2}\right)
\end{equation}
where we have used the fact that the argument of the $\phi$ function is the magnitude of the vector $\mathbf{r}-\frac{a}{2}\left(x+y\right)$, which evaluates to the above. So we can write
\begin{equation}
    \gamma_{xy}(\frac{a}{2}\left(x+y\right)) = -\int d \mathbf{r} x\phi(r) \left(y-\frac{a}{2}\right) \phi\left(\sqrt{\left(x-\frac{a}{2}\right)^2+\left(y-\frac{a}{2}\right)^2+z^2}\right) \Delta U(\mathbf{r})
\end{equation}
Now we want to consider the exponential tractors. Assuming that each $\gamma$ can be written in this form, let's conside the sum
\begin{align}
    \gamma_1\left(e^{i \mathbf{k} \cdot \left(\frac{a}{2}\left(x+y\right)\right)} + e^{i \mathbf{k} \cdot \left(\frac{a}{2}\left(-x-y\right)\right)} + e^{i \mathbf{k} \cdot \left(\frac{a}{2}\left(x-y\right)\right)} + e^{i \mathbf{k} \cdot \left(\frac{a}{2}\left(-x+y\right)\right)}\right)\\
    = \gamma_1e^{\frac{ia}{2} k_x}\left(e^{\frac{ia}{2}k_y}+e^{\frac{ia}{2} \left(-2k_x - k_y\right)} + e^{\frac{ia}{2} \left(-k_y\right)} + e^{\frac{ia}{2} \left(-2k_x + k_y\right)}\right)\\
\end{align}
We use the identity $e^{i\theta}+e^{-i\theta} = 2\cos(\theta)$ to simplify to
\begin{equation}
    \gamma_1e^{\frac{ia}{2} k_x}\left(2\cos\left(\frac{a}{2}k_y\right) + e^{-\frac{2ia}{2}k_x}\left(e^{ia k_y} + e^{-ia k_y}\right)\right) = \gamma_1e^{\frac{ia}{2} k_x}\left(2\cos\left(\frac{a}{2}k_y\right) + e^{-ia k_x}\left(2\cos\left(a k_y\right)\right)\right)
\end{equation}
\subsection{}
(d) Show that all three bands are degenerate at $\mathbf{k}=\mathbf{0}$, and that when $\mathbf{k}$ is directed along either a cube axis $(\Gamma \mathrm{X})$ or a cube diagonal $(\Gamma \mathrm{L})$ there is a double degeneracy. Sketch the energy bands (in analogy to Figure 10.6) along these directions.\\
\subsubsection{Solution}

\end{document}
